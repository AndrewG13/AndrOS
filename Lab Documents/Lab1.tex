%%%%%%%%%%%%%%%%%%%%%%%%%%%%%%%%%%%%%%%%%
%
% CMPT 424
% Fall 2021
% Lab One
%
%%%%%%%%%%%%%%%%%%%%%%%%%%%%%%%%%%%%%%%%%

%%%%%%%%%%%%%%%%%%%%%%%%%%%%%%%%%%%%%%%%%
% Short Sectioned Assignment
% LaTeX Template
% Version 1.0 (5/5/12)
%
% This template has been downloaded from: http://www.LaTeXTemplates.com
% Original author: % Frits Wenneker (http://www.howtotex.com)
% License: CC BY-NC-SA 3.0 (http://creativecommons.org/licenses/by-nc-sa/3.0/)
% Modified by Alan G. Labouseur  - alan@labouseur.com
%
% Currently used by Andrew Giardina - 
% Andrew.Giardina1@marist.edu
%%%%%%%%%%%%%%%%%%%%%%%%%%%%%%%%%%%%%%%%%

%----------------------------------------------------------------------------------------
%	PACKAGES AND OTHER DOCUMENT CONFIGURATIONS
%----------------------------------------------------------------------------------------

\documentclass[letterpaper, 10pt,DIV=13]{scrartcl} 

\usepackage[T1]{fontenc} % Use 8-bit encoding that has 256 glyphs
\usepackage[english]{babel} % English language/hyphenation
\usepackage{amsmath,amsfonts,amsthm,xfrac} % Math packages
\usepackage{sectsty} % Allows customizing section commands
\usepackage{graphicx}
\usepackage[lined,linesnumbered,commentsnumbered]{algorithm2e}
\usepackage{listings}
\usepackage{parskip}
\usepackage{lastpage}

\allsectionsfont{\normalfont\scshape} % Make all section titles in default font and small caps.

\usepackage{fancyhdr} % Custom headers and footers
\pagestyle{fancyplain} % Makes all pages in the document conform to the custom headers and footers

\fancyhead{} % No page header - if you want one, create it in the same way as the footers below
\fancyfoot[L]{} % Empty left footer
\fancyfoot[C]{} % Empty center footer
\fancyfoot[R]{page \thepage\ of \pageref{LastPage}} % Page numbering for right footer

\renewcommand{\headrulewidth}{0pt} % Remove header underlines
\renewcommand{\footrulewidth}{0pt} % Remove footer underlines
\setlength{\headheight}{13.6pt} % Customize the height of the header

\numberwithin{equation}{section} % Number equations within sections (i.e. 1.1, 1.2, 2.1, 2.2 instead of 1, 2, 3, 4)
\numberwithin{figure}{section} % Number figures within sections (i.e. 1.1, 1.2, 2.1, 2.2 instead of 1, 2, 3, 4)
\numberwithin{table}{section} % Number tables within sections (i.e. 1.1, 1.2, 2.1, 2.2 instead of 1, 2, 3, 4)

\setlength\parindent{0pt} % Removes all indentation from paragraphs.

\binoppenalty=3000
\relpenalty=3000

%----------------------------------------------------------------------------------------
%	TITLE SECTION
%----------------------------------------------------------------------------------------

\newcommand{\horrule}[1]{\rule{\linewidth}{#1}} % Create horizontal rule command with 1 argument of height

\title{	
   \normalfont \normalsize 
   \textsc{CMPT 424 - Fall 2021 - Dr. Labouseur} \\[10pt] % Header stuff.
   \horrule{0.5pt} \\[0.25cm] 	% Top horizontal rule
   \huge Lab One  \\     	    % Assignment title
   \horrule{0.5pt} \\[0.25cm] 	% Bottom horizontal rule
}

\author{Andrew Giardina \\ \normalsize Andrew.Giardina1@Marist.edu}

\date{\normalsize\today} 	% Today's date.

\begin{document}
\maketitle % Print the title

%----------------------------------------------------------------------------------------
%   start Question ONE
%----------------------------------------------------------------------------------------
\section{Question One}

\textit{What are	the	advantages	and	disadvantages	of	using	the	same	system	call	interface for	manipulating	both	files	and	devices?}

Being able to utilize the same system call interface for files and devices can provide further simplicity for novice users who simply need to accomplish some task without being bothered by more intricate knowledge of the operating system. Considering that some of the file \& device management system calls behave similarly (request \& release for devices, open \& close for files), it makes sense to pair these up under the same interface (Pages 63-64).

However, there are disadvantages to novelty. One being that files \& devices can appear to be indistinguishable; alike both in representation (how the user is mentally interpreting them) and how they are handled, even though the underlying system calls performed are different (Page 65). This could be at the disadvantage to the user if attempting to replicate their file/device management on a different operating system (since operating system interfaces vary). Thus, the core concept of what \textbf{exactly} is going on is misguided.

%----------------------------------------------------------------------------------------
%   start Question TWO
%----------------------------------------------------------------------------------------
\section{Question Two}

\textit{Would	it	be	possible	for	the	user	to	develop	a	new	command	interpreter	using	the	
system	call	interface	provided	by	the	operating	system?	How?
}

Yes, the user can achieve this.  It can be done by creating their own shell program, and within it, have files named after its specific commands (such as the 'rm' or 'cp' UNIX commands) that will run when the user inputs such values into the CLI. So long as the user-created command interpreter "communicates properly" with the OS's system call interface (follow its set rules/patterns), its functionality should work (Page 52).


\end{document}
