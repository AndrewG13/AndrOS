%%%%%%%%%%%%%%%%%%%%%%%%%%%%%%%%%%%%%%%%%
%
% CMPT 424
% Fall 2021
% Lab Two
%
%%%%%%%%%%%%%%%%%%%%%%%%%%%%%%%%%%%%%%%%%

%%%%%%%%%%%%%%%%%%%%%%%%%%%%%%%%%%%%%%%%%
% Short Sectioned Assignment
% LaTeX Template
% Version 1.0 (5/5/12)
%
% This template has been downloaded from: http://www.LaTeXTemplates.com
% Original author: % Frits Wenneker (http://www.howtotex.com)
% License: CC BY-NC-SA 3.0 (http://creativecommons.org/licenses/by-nc-sa/3.0/)
% Modified by Alan G. Labouseur  - alan@labouseur.com
%
% Currently used by Andrew Giardina - 
% Andrew.Giardina1@marist.edu
%%%%%%%%%%%%%%%%%%%%%%%%%%%%%%%%%%%%%%%%%

%----------------------------------------------------------------------------------------
%	PACKAGES AND OTHER DOCUMENT CONFIGURATIONS
%----------------------------------------------------------------------------------------

\documentclass[letterpaper, 10pt,DIV=13]{scrartcl} 

\usepackage[T1]{fontenc} % Use 8-bit encoding that has 256 glyphs
\usepackage[english]{babel} % English language/hyphenation
\usepackage{amsmath,amsfonts,amsthm,xfrac} % Math packages
\usepackage{sectsty} % Allows customizing section commands
\usepackage{graphicx}
\usepackage[lined,linesnumbered,commentsnumbered]{algorithm2e}
\usepackage{listings}
\usepackage{parskip}
\usepackage{lastpage}

\allsectionsfont{\normalfont\scshape} % Make all section titles in default font and small caps.

\usepackage{fancyhdr} % Custom headers and footers
\pagestyle{fancyplain} % Makes all pages in the document conform to the custom headers and footers

\fancyhead{} % No page header - if you want one, create it in the same way as the footers below
\fancyfoot[L]{} % Empty left footer
\fancyfoot[C]{} % Empty center footer
\fancyfoot[R]{page \thepage\ of \pageref{LastPage}} % Page numbering for right footer

\renewcommand{\headrulewidth}{0pt} % Remove header underlines
\renewcommand{\footrulewidth}{0pt} % Remove footer underlines
\setlength{\headheight}{13.6pt} % Customize the height of the header

\numberwithin{equation}{section} % Number equations within sections (i.e. 1.1, 1.2, 2.1, 2.2 instead of 1, 2, 3, 4)
\numberwithin{figure}{section} % Number figures within sections (i.e. 1.1, 1.2, 2.1, 2.2 instead of 1, 2, 3, 4)
\numberwithin{table}{section} % Number tables within sections (i.e. 1.1, 1.2, 2.1, 2.2 instead of 1, 2, 3, 4)

\setlength\parindent{0pt} % Removes all indentation from paragraphs.

\binoppenalty=3000
\relpenalty=3000

%----------------------------------------------------------------------------------------
%	TITLE SECTION
%----------------------------------------------------------------------------------------

\newcommand{\horrule}[1]{\rule{\linewidth}{#1}} % Create horizontal rule command with 1 argument of height

\title{	
   \normalfont \normalsize 
   \textsc{CMPT 424 - Fall 2021 - Dr. Labouseur} \\[10pt] % Header stuff.
   \horrule{0.5pt} \\[0.25cm] 	% Top horizontal rule
   \huge Lab Two  \\     	    % Assignment title
   \horrule{0.5pt} \\[0.25cm] 	% Bottom horizontal rule
}

\author{Andrew Giardina \\ \normalsize Andrew.Giardina1@Marist.edu}

\date{\normalsize\today} 	% Today's date.

\begin{document}
\maketitle % Print the title

%----------------------------------------------------------------------------------------
%   start Question ONE
%----------------------------------------------------------------------------------------
\section*{Main Question}

\textit{How	is	our	console	like	the	ancient	TTY	subsystem	in	Unix?}

Our operating system console shares several similarities with the TTY subsystem, one of which involves user convenience. So far, our console provides features such as deleting input via the Backspace key, recalling commands using the Up \& Down arrows, and clearing the entire CLI using the 'cls' command. TTYs had these functions as well, being implemented either within its line discipline's cooked mode, or through the actual application running on the system. Another, more important feature comparable to the TTY is how inputs and interrupts are handled. Both have specific programs, our Keyboard Driver and the UART Driver, that handle how characters, numbers, and other text symbols should act and be displayed. Both share the concept of hosting an interrupt queue, in which when an interrupt occurs, is handled by the respective driver before being delivered to the shell once the special Enter key is pressed.  The shell determines whether what it received is even valid, and if so, assesses the command accordingly. As noted in the article \footnote{Article Referenced: https://www.linusakesson.net/programming/tty/}, \textit{"TTY driver is a passive object. It has some data fields and some methods, but the only way it can actually do something is when one of its methods gets called from the context of a process or a kernel interrupt handler."} We can see this same relationship in the physical layout of our OS project, where the only time the Keyboard Driver is in use or active is at the will of the kernel. Finally, something to prospect for the future of our project (without looking ahead!) is whether our OS will be able to handle multiple user processes at once.  The TTY article addresses session management, accessing each running program, and halting \& terminating them if need be. These are properties supplied by the TTY Driver. While I am sure any programs the user will write in our OS (User input textarea) can be manipulated or ran in the background, I wonder if we will be building out the ability to have several programs ran, tracked, and managed in the background in conjunction with one another!  

\end{document}